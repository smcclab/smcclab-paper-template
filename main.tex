\documentclass[manuscript, screen, review, nonacm]{acmart}

\usepackage{multicol}
% Suppress fancyhdr header commands used in the CHIZen checklist
% (acmart already loads fancyhdr, so we redefine rather than define)
\AtBeginDocument{%
  \renewcommand{\rhead}[1]{}%
  \renewcommand{\lhead}[1]{}%
}

\setcopyright{cc}
\copyrightyear{2026}
\acmYear{2026}
\acmDOI{}
\acmConference[]{}{}
\acmISBN{}

\begin{document}

\title{SMCClab Paper Writing Template and Guide}

\author{First Author}
\affiliation{
  \institution{Australian National University}
  \city{Canberra}
  \country{Australia}
}
\email{first.author@anu.edu.au}

\author{Second Author}
\affiliation{
  \institution{Australian National University}
  \city{Canberra}
  \country{Australia}
}
\email{second.author@anu.edu.au}

\begin{abstract}
  The abstract should be a concise (150--200 word) summary of the paper. It should state the
  motivation and problem being addressed, the approach or method used, the key findings, and
  the contribution or significance of the work. Write the abstract last, once the rest of the
  paper is complete.
\end{abstract}

\keywords{human-computer interaction, keyword two, keyword three}

\maketitle

\section{Introduction}

The introduction motivates the work and situates it within the broader research context.
It should establish why the problem matters, what gap in existing knowledge this paper
addresses, and briefly indicate the approach taken. Close the introduction with a clear
statement of the paper's contributions and a roadmap of the remaining sections
\cite{caine_sample_size:2016}.

\section{Related Work}

Review the prior work most relevant to this paper. Organise related work thematically
rather than as an annotated list. Identify what has been done, what gaps remain, and how
this work builds on or differs from existing research. Related work should justify the
design decisions and evaluation approach described later.

\section{Method}

Describe the study or system design in sufficient detail for replication. Include
participant recruitment and demographics, apparatus or materials, procedure, and any
ethical considerations. For design research, describe the design process and rationale.
For empirical studies, state hypotheses or research questions and justify the chosen
methodology.

\subsection{Participants}

Describe who participated, how they were recruited, inclusion and exclusion criteria,
and relevant demographic information. Report the sample size and justify it with
reference to appropriate norms for the methodology used \cite{caine_sample_size:2016}.

\subsection{Apparatus and Materials}

Describe the hardware, software, stimuli, or instruments used. Include version numbers
and configuration details where relevant. If a custom system was built, provide enough
detail to understand its operation; a full technical description can go in a separate
system description section if needed.

\subsection{Procedure}

Describe the sequence of events participants experienced, from consent through to
debrief. Include timing, task descriptions, and any counterbalancing or randomisation
applied.

\section{Results}

Present findings without interpretation. For quantitative studies, report descriptive
statistics followed by inferential tests, including effect sizes and confidence intervals.
For qualitative studies, present themes with illustrative quotes or examples. Use figures
and tables to support the narrative rather than repeat it.

\section{Discussion}

Interpret the results in relation to the research questions and prior work. Explain what
the findings mean, why they occurred, and what implications they have for design or
theory. Acknowledge limitations of the study and suggest directions for future work.

\section{Conclusion}

Summarise the key contributions of the paper in two or three paragraphs. Restate the
problem, the approach, the main findings, and the significance of the work. Do not
introduce new material. End with a forward-looking statement about impact or next steps.

\bibliographystyle{ACM-Reference-Format}
\bibliography{references}

\input{components/CHIZen-Paper-Checklist}

\end{document}
